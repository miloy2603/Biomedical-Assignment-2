\documentclass[a4paper,12pt]{report}
\usepackage{graphicx}
\title{EVOLUTION OF MODERN HEALTHCARE SYSTEM}
\author{NAME :- MILOY KUMAR MANDAL\\ROLL NO :- 21111032\\BRANCH :- BIOMEDICAL ENGINEERING\\SECTION:-A\\SEMESTER:- 1st\\YEAR:-1st}
\date{\today}
\clearpage
\begin{document}
\maketitle
\pagenumbering{roman}
\tableofcontents
\newpage
\pagenumbering{arabic}

\chapter{THE EVOLUTION OF MODERN HEALTHCARE SYSTEM}

\section{Introduction}
In the past, hospital administrators have concentrated on meeting their budget and staying out of the newspapers. Over the last 10 years, however,there has been a shift to concentrating on patient safety in hospitals. This chapter will first examine hospital patient safety in a historical context, explaining how the patient safety anomaly is related to a 19th century hospital construct that is no longer appropriate for an increasingly at-risk group of patients. The final section of the chapter will concentrate on the emergence of systems to improve patient safety in acute care hospitals.

\section{Historical Perspective}
In many ways acute care hospitals are designed to deliver health care as it was practiced in the 19th century. The technological advances in medicine of the late 20th century are superimposed on a system originally designed to care for patients admitted largely for bed rest and convalescence.\\

Originally hospitals were charitable institutions established to care for the poor . Apart from performing a limited number of operations, hospitals offered little that could not be provided by a doctor in the home.Medical students learned their craft in acute care hospitals, mainly how they make sense of symptoms and signs to reach a diagnosis.Therapeutic options were few. Medical specialists earned their living in their consulting rooms or by visiting patients in their homes. They went to the hospital only once or twice a week to make rounds, accompanied by their assigned team of
students and doctors-in-training. They gave freely of their time, and in return maintained a profile as a source of patient referrals and benefited from the prestige and sense of charity associated with a teaching hospital appointment.\\

The hospital was constructed around the needs of specialist doctors, who in return for giving their time freely had their own wards, operating  theaters, recovery areas, nursing staff, and medical teams. They visited the hospital and did rounds at their convenience. Patients were cared for within the limits of what was available. Pain relief was possible, but curative drugs were relatively rare. Diagnostic services were limited to simple x-rays and basic blood tests. Intravenous fluid was rarely used. Operations were limited and not supported by the same sophisticated perioperative care we have today. If one was seriously ill, it was more common to call a doctor to come to the patient, rather than call an ambulance to take the patient to a hospital.\\

Around the late 1940s, health care delivery changed and has continued to evolve exponentially to the present day. Antibiotics were developed; drugs controlling cardiovascular and respiratory conditions became available; chemotherapy and radiotherapy were increasingly used for cancer; dialysis and other supportive interventions for chronic conditions became widely available; diagnostic procedures enabled us to image and understand much of the body’s disease processes previously guessed at by external signs  and symptoms; and the number of noninvasive and invasive surgical options expanded.\\

Hospitalized patients are now admitted for cure or at least control of their diseases. The hospital population is older, usually with multiple comorbidities, and often further at risk as a result of the procedures and drugs being used. Expectations of hospitalized patients are high—often unrealistically so—and reinforced by widespread and frequent media reports of wonder drugs and miracle operations with little in the way of balance.People still age and become ill with diseases for which medicine has little or nothing to offer.\\

While the nature of the hospital patient population and its expectations has changed considerably, the system within which they are managed has evolved little since the 19th century. Patients in emergency departments are still processed in the same way. Patients are still “owned” by a single specialist doctor, and most of the day-to-day activities are supported by doctors-in-training. Nursing staff still records vital signs manually, with little or no power to act on abnormalities. Consultant physicians who are ultimately responsible for the patient’s care still largely manage from a distance.\\

What may have worked well in the 19th century does not necessarily guarantee safe management in the 21st century. Specialization may not equip consultants with the skills, knowledge, and experience necessary to care for the complex co-morbidities that patients increasingly have, or for when the patient becomes seriously ill.Junior house doctors are either too inexperienced and lack the skills and knowledge to care for complex at-risk patients,or they tend to become too specialized and fail to receive adequate training in the other areas that are necessary to treat complex patients, especially those who become seriously ill. Silos, or vertical structures within hospitals such as wards, units, and departments, are well developed in acute care hospitals, but there is a paucity of horizontal system integration across the silos.\\

While the silos adequately manage the specialized component of a patient’s condition, they usually prove inadequate for co-existing conditions and for patient complications. The hospital usually does not provide the necessary systems, or horizontal connections, to support the vertical silos.\\

It is not surprising therefore that there are many potentially preventable deaths and serious adverse events in acute care hospitals . Moreover, many of these potentially preventable deaths are preceded for many hours by a slow deterioration in vital signs. For example, up to 90percent of hospital cardiac arrests are preceded by relatively slow and potentially reversible deterioration . Admissions of patients to the intensive care unit (ICU) from the general wards are often preceded by the same predictable slow deterioration . If we are concerned about a seriously ill patient in the community, we call an ambulance. If we have a similar patient in an acute
care hospital, it appears we have little in the way of systematic intervention. Nurses record abnormal findings; junior doctors may be informed about the patient in a hierarchical way—the most junior first, with information passed up the line, depending on the level of understanding and awareness of how serious the patient’s condition may be. Alternatively, in non-teaching hospitals, the nurse would first contact the patient’s primary physician, who might not even be on-site. If unable to attend to the patient,this person may request a consultation by someone more available or expert. Sometimes the patient may be referred to an acute care physician, such as an intensivist. Thus, the response to the crisis is built in an ad-hoc manner, piece by piece.The only systematic and organized approach is often the cardiac arrest team (a predefined and prepared group of responders
with specialized resources), called after the patient has “died” .\\





\end{document}